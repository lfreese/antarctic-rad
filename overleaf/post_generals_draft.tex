\documentclass[12]{article}
\usepackage[margin=1.0in]{geometry}
\usepackage[utf8]{inputenc}
\usepackage{titlesec}
\usepackage{physics}
\usepackage{graphicx}
\usepackage{siunitx}
\usepackage{cancel}
\usepackage{amsmath}
\usepackage{textcomp}
\usepackage{gensymb}
\usepackage{natbib}
\usepackage{bm}
\usepackage{setspace}
\usepackage[version=4]{mhchem}

\titleformat*{\subsection}{\normalfont\fontfamily{phv}}
%\titleformat*{\subsection}[runin]{}{}{}{}[]
  \titleformat{\subsection}[runin]{\normalfont\bfseries}{\thesubsection.}{3pt}{}
  \titleformat{\subsubsection}[runin]{\normalfont\bfseries}{\thesubsubsection.}{3pt}{}

\title{{\textsc{\Large Antarctic Radiative and Temperature responses to a doubling of $\text{CO}_2$}}}
\author{\textsc{Lyssa Freese}
\\\\
Advised by Prof. Tim Cronin}
\doublespacing

\begin{document}
\maketitle
\thispagestyle{empty}

\setlength{\leftskip}{1.1cm}
\setlength{\rightskip}{1.1cm}


\bigskip
\bigskip

{\textsc{Abstract.} }Greenhouse gases (GHGs), such as \ce{CO2}, impact global and local outgoing longwave radiation (OLR). The Antarctic is known for its a strong near-surface temperature inversion, where the addition of GHGs can lead to increased OLR during the due to higher radiating temperatures higher in the atmosphere.  Here we develop a radiative-advective-turbulent single-column model based on observed temperatures at the South Pole and timestep it forward under different \ce{CO2} concentrations. We discuss 1) radiative diagnostics of the observed temperature profiles under varying \ce{CO_2} concentrations and 2) variation of temperature and radiative fluxes as we timestep the model forward to equilibrium under a normal and doubled \ce{CO_2} scenario. We confirm previous results showing negative top-of-atmosphere forcing with increased \ce{CO_2} during all seasons but austral winter. Despite this negative forcing, we find increased temperatures at the surface across all seasons and cooling in the stratosphere with a doubling of \ce{CO_2}.

\section{Introduction}
\section{Methods}
\section{Results}
\section{Discussion}

\bibliographystyle{apalike}
\bibliography{references.bib}

\end{document}
